% Options for packages loaded elsewhere
\PassOptionsToPackage{unicode}{hyperref}
\PassOptionsToPackage{hyphens}{url}
%
\documentclass[
  ignorenonframetext,
]{beamer}
\usepackage{pgfpages}
\setbeamertemplate{caption}[numbered]
\setbeamertemplate{caption label separator}{: }
\setbeamercolor{caption name}{fg=normal text.fg}
\beamertemplatenavigationsymbolsempty
% Prevent slide breaks in the middle of a paragraph
\widowpenalties 1 10000
\raggedbottom
\setbeamertemplate{part page}{
  \centering
  \begin{beamercolorbox}[sep=16pt,center]{part title}
    \usebeamerfont{part title}\insertpart\par
  \end{beamercolorbox}
}
\setbeamertemplate{section page}{
  \centering
  \begin{beamercolorbox}[sep=12pt,center]{part title}
    \usebeamerfont{section title}\insertsection\par
  \end{beamercolorbox}
}
\setbeamertemplate{subsection page}{
  \centering
  \begin{beamercolorbox}[sep=8pt,center]{part title}
    \usebeamerfont{subsection title}\insertsubsection\par
  \end{beamercolorbox}
}
\AtBeginPart{
  \frame{\partpage}
}
\AtBeginSection{
  \ifbibliography
  \else
    \frame{\sectionpage}
  \fi
}
\AtBeginSubsection{
  \frame{\subsectionpage}
}
\usepackage{amsmath,amssymb}
\usepackage{iftex}
\ifPDFTeX
  \usepackage[T1]{fontenc}
  \usepackage[utf8]{inputenc}
  \usepackage{textcomp} % provide euro and other symbols
\else % if luatex or xetex
  \usepackage{unicode-math} % this also loads fontspec
  \defaultfontfeatures{Scale=MatchLowercase}
  \defaultfontfeatures[\rmfamily]{Ligatures=TeX,Scale=1}
\fi
\usepackage{lmodern}
\ifPDFTeX\else
  % xetex/luatex font selection
\fi
% Use upquote if available, for straight quotes in verbatim environments
\IfFileExists{upquote.sty}{\usepackage{upquote}}{}
\IfFileExists{microtype.sty}{% use microtype if available
  \usepackage[]{microtype}
  \UseMicrotypeSet[protrusion]{basicmath} % disable protrusion for tt fonts
}{}
\makeatletter
\@ifundefined{KOMAClassName}{% if non-KOMA class
  \IfFileExists{parskip.sty}{%
    \usepackage{parskip}
  }{% else
    \setlength{\parindent}{0pt}
    \setlength{\parskip}{6pt plus 2pt minus 1pt}}
}{% if KOMA class
  \KOMAoptions{parskip=half}}
\makeatother
\usepackage{xcolor}
\newif\ifbibliography
\setlength{\emergencystretch}{3em} % prevent overfull lines
\providecommand{\tightlist}{%
  \setlength{\itemsep}{0pt}\setlength{\parskip}{0pt}}
\setcounter{secnumdepth}{-\maxdimen} % remove section numbering
\usepackage{framed,color}
\pgfdeclareimage[height=1cm, width=2.5cm]{logo}{ChileCompra.png}
\logo{\pgfuseimage{logo}}
\usepackage{wrapfig}
\usepackage{booktabs}
\usepackage{longtable}
\usepackage{array}
\usepackage{multirow}
\usepackage{float}
\usepackage{colortbl}
\usepackage{pdflscape}
\usepackage{threeparttable}
\usepackage{threeparttablex}
\usepackage[normalem]{ulem}
\usepackage{makecell}
\usepackage{xcolor}
\ifLuaTeX
  \usepackage{selnolig}  % disable illegal ligatures
\fi
\IfFileExists{bookmark.sty}{\usepackage{bookmark}}{\usepackage{hyperref}}
\IfFileExists{xurl.sty}{\usepackage{xurl}}{} % add URL line breaks if available
\urlstyle{same}
\hypersetup{
  pdftitle={Resultados Mes 31},
  pdfauthor={Dirección ChileCompra},
  hidelinks,
  pdfcreator={LaTeX via pandoc}}

\title{Resultados Mes 31}
\subtitle{Nueva metodología de Ahorro en Convenio Marco}
\author{Dirección ChileCompra}
\date{2024-07-09}

\begin{document}
\frame{\titlepage}

\begin{frame}[fragile]
\begin{verbatim}
##        convenio ahorro_promedio productos total_ahorro cobertura_monto
## 1           Gas      0.43455019        61   1226821583       94.347590
## 2  Alimentos RM      0.19128074       502    924292701       59.866301
## 3          Aseo      0.29678013        96    635391973       53.496232
## 4 Insumos Salud      0.41319267        29    334352874        9.617997
## 5  Computadores      0.27272712         3    285216764       43.290916
## 6  Combustibles      0.03162037         4    128449862       71.160061
## 7    Ferreteria      0.10836902        96    112918541       23.219391
\end{verbatim}
\end{frame}

\begin{frame}{Contenido}
\protect\hypertarget{contenido}{}
\begin{itemize}
\tightlist
\item
  Resultados de marzo 2024
\end{itemize}

\vspace{0.25cm}

\begin{itemize}
\tightlist
\item
  Comparación con meses previos
\end{itemize}
\end{frame}

\begin{frame}[fragile]{Resultados del mes}
\protect\hypertarget{resultados-del-mes}{}
\begin{verbatim}
## [1] 0
\end{verbatim}

El ahorro total del mes asciende a: \textbf{\$4.094 millones de pesos}
(USD 4,2 MM)

\begin{table}
\centering
\resizebox{\linewidth}{!}{
\begin{tabular}{lcccc}
\toprule
\textbf{Convenio} & \textbf{Prod.Monitoreados} & \textbf{Ahorro Promedio(\%)} & \textbf{Ahorro (\$)} & \textbf{Cobertura*}\\
\midrule
Gas & 67 & 43,5\% & 1.226.821.583 & 100,0\%\\
Alimentos RM & 953 & 18,8\% & 1.023.978.181 & 67,4\%\\
Aseo & 415 & 28,4\% & 729.885.954 & 63,6\%\\
Insumos Salud & 81 & 36,0\% & 391.981.683 & 13,0\%\\
Computadores & 6 & 28,5\% & 355.184.205 & 50,1\%\\
Ferreteria & 406 & 13,3\% & 237.463.550 & 37,9\%\\
Combustibles & 4 & 3,2\% & 128.341.836 & 71,2\%\\
\bottomrule
\end{tabular}}
\end{table}

\vspace{0.5cm}

\footnotesize

\textcolor{blue}{Total transado del mes} = \$23.214 millones de pesos
(USD24 MM)

\tiny

*Considera todas las transacciones de los 8 convenios monitoreados.
\center \footnotesize \textcolor{blue}{Cobertura agregada del mes} =
56,8\%
\end{frame}

\begin{frame}{Comparación con meses previos}
\protect\hypertarget{comparaciuxf3n-con-meses-previos}{}
\begin{block}{Ahorro}
\protect\hypertarget{ahorro}{}
\footnotesize

\includegraphics[width=0.98\linewidth,height=0.65\textheight]{reporte_ahorro_files/figure-beamer/unnamed-chunk-4-1}
\end{block}
\end{frame}

\begin{frame}{Comparación con meses previos}
\protect\hypertarget{comparaciuxf3n-con-meses-previos-1}{}
\begin{block}{Cobertura}
\protect\hypertarget{cobertura}{}
\footnotesize

\includegraphics[width=0.97\linewidth,height=0.6\textheight]{reporte_ahorro_files/figure-beamer/unnamed-chunk-5-1}
\end{block}
\end{frame}

\begin{frame}{Notas}
\protect\hypertarget{notas}{}
\begin{itemize}
\tightlist
\item
  \textbf{Combustibles}: Se han actualizado las cifras desde el mes de
  julio de 2023 en adelante.
\item
  \textbf{Gas}: Se actualizó el cálculo a partir de la misma fecha
  utilizando precios históricos de call centers, repasando cifras
  presentadas previamente construidas en base a estimaciones.
\end{itemize}
\end{frame}

\end{document}
